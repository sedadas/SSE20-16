\section{Pentesting - Redteaming}

\subsection{It's A Leak!}
The first thing I did was to look for useful information on social media. After a bit of searching I found the Twitter account \href{https://twitter.com/notbotstudios}{@NotbotStudios}, which seems to be the official account of the company. On April 8th this account tweeted an \href{https://twitter.com/NotbotStudios/status/1247844703162294272}{image}, which shows the laptop screen of one of the IT employees of the film studio. On the left side of this image, I spotted a \href{https://www.dropbox.com/s/cmopqfl4iwpfwz5/}{Dropbox URL}, which leads to a password protected ZIP file called "templates.zip". \\

On April 4th there was another interesting tweet by the company account, which mentions the personal twitter account of the newly hired IT-Support employee Lisa Boehm, \href{https://twitter.com/lboehm1985}{@lboehm1985}. I found Lisa's \href{https://twitter.com/lboehm1985/status/1248692516083445761}{latest tweet} very interesting, because it contains a picture of her work TODO list where the last line is blackened out. After downloading the image I was able to make the last line readable again, by editing the brightness and contrast. The text she tried to censor is "Default PW: Ar513ximTtz4!". Of course it's always a bad idea to publicly post an image of confidential work information, but not properly censoring out a password is especially careless. \\

Using the password "Ar513ximTtz4!" I unlocked the templates.zip file and found a python script that seems to be used internally to generate an automated email to send new employees their initial E-Mail and Intranet account passwords. The initial passwords are generated by hashing together multiple strings, including the first and last name of the employee, the current date and a string called "secure\_seed". This seed has already been discovered in a previous Pentest multiple years ago ("secure\_seed=\allowbreak omebsAupGzaYgggl3h44DiO4dARxjZe8") and seems to not have been changed since then, which ironically makes it not very secure. \\

Next I changed the source code of the python script slightly (\hyperref[code:generate_script]{see listing~\ref*{code:generate_script} on page~\pageref*{code:generate_script}}), so it doesn't use the current date, but lets me specify the date as a command line argument instead. Thanks to Lisa's Twitter I knew that she got the job at Notbot Studios on March 24th and from the previously mentioned tweet on April 4th I knew that she was already on board with the Team for a few days at that point. Therefore I knew that her initial Intranet password must have been generated in the few days between those two dates. Trying the script with March 30th (Monday of her first work week) I was able to generate the password "285cb161deeac78c873117b96ac464af" and username "lboehm". With these login details I successfully logged into the Intranet website, which means that Lisa still hasn't changed her password, even though the generated welcome email specifically says to change the initial password immediately.

\lstinputlisting[caption=Edited Python script,label=code:generate_script,style=c,language=Python]{generate_edited.py}

\subsection{Spoil[er]ed By Bad Movies}

The Intranet website contains a file upload form, where you can upload a movie script as a PDF file. The uploaded PDF is run through Apache Tika to extract metadata and this metadata is then displayed on the webpage in a small test report. On the previously mentioned TODO list from Lisa Boehm I noticed that one of the entries was to upgrade Tika from version 1.12 to a newer version. Assuming that this update has not been done yet, I searched for possible vulnerabilities and quickly found \href{https://www.cvedetails.com/cve/CVE-2016-4434/}{CVE-2016-4434}, which is a vulnerability to XML External Entity (XXE) attacks via XMP metadata in PDF files and other file formats. \\

To exploit this vulnerability, I first populated a blank PDF file with a lot of XMP metadata, to see which fields show up in the report on the website and can therefore be used to retrieve data from the server. Next I declared an XML external entity that contains the path \textit{/etc/passwd} in the DOCTYPE element and placed this entity inside the "dc:description" entity (\hyperref[code:xxe]{see listing~\ref*{code:xxe} on page~\pageref*{code:xxe}}), because this description is then displayed in the metadata report. Uploading this crafted PDF reveals the contents of the \textit{/etc/passwd} file. \\

The last user in \textit{/etc/passwd}, called "sandbox", was interesting to me because it seems to be a user account with a shell and a normal home directory. I used the same payload as before but switched the "file:///etc/passwd" to "file:///home/sandbox/" to view the contents of the home directory. Doing the same thing to check further files, I found the file \textit{id\_rsa} inside the \textit{/home/sandbox/.ssh/} directory, which is a OpenSSH private key. With this key I was able to successfully login to the sandbox user via SSH. \\

\begin{lstlisting}[caption=XMP metadata payload to retrieve /etc/passwd,label=code:xxe,style=simple]
2 0 obj
<<
/Type /Metadata
/Subtype /XML
/Length 787
>>
stream
<?xpacket begin="" id="W5M0MpCehiHzreSzNTczkc9d"?>
<!DOCTYPE foo [
    <!ENTITY xxe SYSTEM "file:///etc/passwd">
]>
<x:xmpmeta xmlns:x="adobe:ns:meta/">
<rdf:RDF xmlns:rdf="http://www.w3.org/1999/02/22-rdf-syntax-ns#">
    <rdf:Description xmlns:dc="http://purl.org/dc/elements/1.1/">
        <dc:title>
            <rdf:Alt>
                <rdf:li xml:lang="x-default">Blank PDF Document</rdf:li>
            </rdf:Alt>
        </dc:title>
        <dc:format>application/pdf</dc:format>
        <dc:description>
            <rdf:Alt>
                <rdf:li>&xxe;</rdf:li>
            </rdf:Alt>
        </dc:description>
    </rdf:Description>
    <rdf:Description xmlns:xmpRights="http://ns.adobe.com/xap/1.0/rights/"/>
</rdf:RDF>
</x:xmpmeta>
<?xpacket end="w"?>
endstream
endobj
\end{lstlisting}
