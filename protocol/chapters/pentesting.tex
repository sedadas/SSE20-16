\section{Pentesting - Redteaming}

\subsection{It's A Leak!}
The first thing I did was to look for useful information on social media. After a bit of searching I found the Twitter account \href{https://twitter.com/notbotstudios}{@NotbotStudios}, which seems to be the official account of the company. On April 8th this account tweeted an \href{https://twitter.com/NotbotStudios/status/1247844703162294272}{image}, which shows the laptop screen of one of the IT employees of the film studio. On the left side of this image, I spotted a Dropbox URL \href{https://twitter.com/NotbotStudios/status/1247844703162294272}, which leads to a password protected ZIP file called "templates.zip". \\

On April 4th there was another interesting tweet by the company account, which mentions the personal twitter account of the newly hired IT-Support employee Lisa Boehm, \href{https://twitter.com/lboehm1985}{@lboehm1985}. I found Lisa's \href{https://twitter.com/lboehm1985/status/1248692516083445761}{latest tweet} very interesting, because it contains a picture of her work TODO list where the last line is blackened out. After downloading the image I was able to make the last line readable again, by editing the brightness and contrast. The text she tried to censor is "Default PW: Ar513ximTtz4!". Of course it's always a bad idea to publicly post an image of confidential work information, but not properly censoring out a password is especially careless. \\

Using the password "Ar513ximTtz4!" I unlocked the templates.zip file and found a python script that seems to be used internally to generate an automated email to send new employees their initial E-Mail and Intranet account passwords. The initial passwords are generated by hashing together multiple strings, including the first and last name of the employee, the current date and a string called "secure\_seed". This seed has already been discovered in a previous Pentest multiple years ago ("secure\_seed=\allowbreak omebsAupGzaYgggl3h44DiO4dARxjZe8") and seems to not have been changed since then, which ironically makes it not very secure. \\

Next I changed the source code of the python script slightly (\hyperref[code:generate_script]{see listing~\ref*{code:generate_script} on page~\pageref*{code:generate_script}}), so it doesn't use the current date, but lets me specify the date as a command line argument instead. Thanks to Lisa's Twitter I knew that she got the job at Notbot Studios on March 24th and from the previously mentioned tweet on April 4th I knew that she was already on board with the Team for a few days at that point. Therefore I knew that her initial Intranet password must have been generated in the few days between those two dates. Trying the script with March 30th (Monday of her first work week) I was able to generate the password "285cb161deeac78c873117b96ac464af" and username "lboehm". With these login details I successfully logged into the Intranet website, which means that Lisa still hasn't changed her password, even though the generated welcome email specifically says to change the initial password immediately.

\lstinputlisting[caption=Edited Python script,label=code:generate_script,style=c,language=Python]{generate_edited.py}