%%%%%%%%%%%%%%%%%%%%%%%%%%%%%%%%%%%%%%%%%%%%%%%%%%%%%%%%%%%%%%%%%%%%%%
%
% Institut für Rechnergestuetzte Automation
% Forschungsgruppe Industrial Software
% Arbeitsgruppe ESSE
% https://security.inso.tuwien.ac.at/
% lva.security@inso.tuwien.ac.at
%
%%%%%%%%%%%%%%%%%%%%%%%%%%%%%%%%%%%%%%%%%%%%%%%%%%%%%%%%%%%%%%%%%%%%%%

\documentclass[12pt,a4paper,titlepage,oneside]{scrartcl}
\newcommand{\lang}{en}
\usepackage{esseProtocol}

%%%%%%%%%%%%%%%%%%%%%%%%%%%%%%%%%%%%%%%%%%%%%%%%%%%%%%%%%%%%%%%%%%%%%%
%
% FOR STUDENTS
%
%%%%%%%%%%%%%%%%%%%%%%%%%%%%%%%%%%%%%%%%%%%%%%%%%%%%%%%%%%%%%%%%%%%%%%

% Team number or "0" for Lab0
%TODO team number
\newcommand{\team}{XX}
% Date
%TODO fill in creation date
\newcommand{\datum}{01.01.1970}
%TODO lab number
% valid values: "Lab0", "Lab1" (be sure to use Uppercase for first character)
\newcommand{\lab}{LabX}

%TODO name of course
\newcommand{\lvaname}{LVA-Name}
%TODO number of course
\newcommand{\lvanr}{183.XXX}
%TODO year and term, for example: "SS 2012", "WS 2012", "SS 2013", etc.
\newcommand{\semester}{WS/SS 20XX}

% Student data in Lab0 or 1. student of team in Lab1
\newcommand{\studentAName}{John Doe}
\renewcommand{\studentAMatrnr}{4711081}

% 2. student of team in Lab1, for Lab0 or if your team has less students, remove these 2 lines
\newcommand{\studentBName}{Martha Musterfrau}
\renewcommand{\studentBMatrnr}{1234567}

% 3. student of team in Lab1, for Lab0 or if your team has less students, remove these 2 lines
\newcommand{\studentCName}{Otto Mustermann}
\renewcommand{\studentCMatrnr}{0815421}

% 4. student of team in Lab1, for Lab0 or if your team has less students, remove these 2 lines
\newcommand{\studentDName}{Otto Mustermann}
\renewcommand{\studentDMatrnr}{0995421}

% 5. student of team in Lab1, for Lab0 or if your team has less students, remove these 2 lines
\newcommand{\studentEName}{Otto Mustermann}
\renewcommand{\studentEMatrnr}{0236214}

%%%%%%%%%%%%%%%%%%%%%%%%%%%%%%%%%%%%%%%%%%%%%%%%%%%%%%%%%%%%%%%%%%%%%%
%
% DO NOT CHANGE THE FOLLOWING PART
%
%%%%%%%%%%%%%%%%%%%%%%%%%%%%%%%%%%%%%%%%%%%%%%%%%%%%%%%%%%%%%%%%%%%%%%

\newcommand{\colormode}{color}
\newcommand{\dokumenttyp}{Report \lab}

\begin{document}

\maketitle
\setcounter{section}{0}
\setcounter{tocdepth}{2}
\tableofcontents

%%%%%%%%%%%%%%%%%%%%%%%%%%%%%%%%%%%%%%%%%%%%%%%%%%%%%%%%%%%%%%%%%%%%%%
%
% CONTENT OF DOCUMENT STARTS HERE
%
%%%%%%%%%%%%%%%%%%%%%%%%%%%%%%%%%%%%%%%%%%%%%%%%%%%%%%%%%%%%%%%%%%%%%%

\section{Section 1}

\subsection{Notes}
\emph{Notes:}
\begin{itemize}
    \item Either use this english version or the german version in \lstinline{protokoll.tex}
    \item Replace all variables below \emph{FOR STUDENTS} in this .tex file
    \item Replace the placeholders for your name and your student id (MatNr)
    \item Delete this hint and example chapters before the final submission
    \item Keep an eye on the required file formats and \emph{file names}
    \item Execute \lstinline{pdflatex} at least twice, in order to get correct references and page numbers in the pdf document
    \item You may also use the makefile for this purpose: \lstinline{make en}
\end{itemize}

\subsection{Sub-Section 1}
Lorem ipsum dolor sit amet, consetetur sadipscing elitr, sed diam nonumy eirmod tempor invidunt ut labore et dolore magna aliquyam erat, sed diam voluptua. At vero eos et accusam et justo duo dolores et ea rebum. Stet clita kasd gubergren, no sea takimata sanctus est Lorem ipsum dolor sit amet. Lorem ipsum dolor sit amet, consetetur sadipscing elitr, sed diam nonumy eirmod tempor invidunt ut labore et dolore magna aliquyam erat, sed diam voluptua. At vero eos et accusam et justo duo dolores et ea rebum. Stet clita kasd gubergren, no sea takimata sanctus est Lorem ipsum dolor sit amet.

\subsection{Sub-Section 2}
Lorem ipsum dolor sit amet, consetetur sadipscing elitr, sed diam nonumy eirmod tempor invidunt ut labore et dolore magna aliquyam erat, sed diam voluptua. At vero eos et accusam et justo duo dolores et ea rebum. Stet clita kasd gubergren, no sea takimata sanctus est Lorem ipsum dolor sit amet. Lorem ipsum dolor sit amet, consetetur sadipscing elitr, sed diam nonumy eirmod tempor invidunt ut labore et dolore magna aliquyam erat, sed diam voluptua. At vero eos et accusam et justo duo dolores et ea rebum. Stet clita kasd gubergren, no sea takimata sanctus est Lorem ipsum dolor sit amet.

\section{Section 2}

\subsection{Sub-Section 1}
Lorem ipsum dolor sit amet, consetetur sadipscing elitr, sed diam nonumy eirmod tempor invidunt ut labore et dolore magna aliquyam erat, sed diam voluptua. 

\subsection{Sub-Section 2}
Lorem ipsum dolor sit amet, consetetur sadipscing elitr, sed diam nonumy eirmod tempor invidunt ut labore et dolore magna aliquyam erat, sed diam voluptua. At vero eos et accusam et justo duo dolores et ea rebum. 

\subsection{Sub-Section 3}
Lorem ipsum dolor sit amet, consetetur sadipscing elitr, sed diam nonumy eirmod tempor invidunt ut labore et dolore magna aliquyam erat, sed diam voluptua. 

\section{Demos}

\subsection{Source Code format}
In these sections a few examples how to format source code in \LaTeX are given.

(\hyperref[code:example1]{see listing~\ref*{code:example1} on page~\pageref*{code:example1}} and \hyperref[code:example2]{see listing~\ref*{code:example2} on page~\pageref*{code:example2}}).

You can include short code snippets or commands directly inline with \lstinline{lstinline block}.

\lstinputlisting[caption=Example C/C++ file,label=code:example1,style=c]{example.c}

\begin{lstlisting}[caption=Example bash script,label=code:example2,style=simple]
#!/bin/bash
echo "Bash version ${BASH_VERSION}..."
for i in {0..10..2}
  do
     echo "Welcome $i times"
 done

echo "some very very very very very very very very very very very very very very very very very very very very long string"

exit 0;
\end{lstlisting}

\subsection{Images}

Here is an example how to insert an image into this document.
(\hyperref[fig:logo1]{see figure~\ref*{fig:logo1} on page~\pageref*{fig:logo1}}).

\begin{figure}[h!]
  \centering
  \fbox{
    \includegraphics[width=0.4\textwidth]{./imgs/logos/esse-logo-color.png}
  }
  \caption{ESSE Logo}
  \label{fig:logo1}
\end{figure}


%%%%%%%%%%%%%%%%%%%%%%%%%%%%%%%%%%%%%%%%%%%%%%%%%%%%%%%%%%%%%%%%%%%%%%
%
% DO NOT CHANGE THE FOLLOWING PART
%
%%%%%%%%%%%%%%%%%%%%%%%%%%%%%%%%%%%%%%%%%%%%%%%%%%%%%%%%%%%%%%%%%%%%%%

\end{document}


